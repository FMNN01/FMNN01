\documentclass[a4paper,12pt]{article}
\usepackage{float,amsfonts,amsthm,amsmath,amssymb,wasysym,graphicx,tikz,textcomp,pgf,mathrsfs,verbatim, graphicx, ifthen, enumitem, pifont,fancyhdr, blindtext, lipsum,xcolor,hyperref,fixmath}
\usepackage[makeroom]{cancel}
\usepackage{amsfonts}
\usepackage{amsmath}
\usepackage{amssymb}
\usepackage{amsthm}

\usepackage[english]{babel}
\usepackage[T1]{fontenc}
\usepackage[utf8]{inputenc}


\newcommand{\argmax}{\mathrm{argmax}}
\newcommand{\rank}{\mathrm{rank}}

\newtheorem{proposition}{Proposition}


\begin{document}

\title{Homework 1 \\ NUMA11}
\author{
  Karl \textsc{Lind\'{e}n} \\
  <karl.linden.887@student.lu.se> \\
  Oscar \textsc{Nilsson} \\
  <erik-oscar-nilsson@live.se>
}

\maketitle
\clearpage

\subsection*{Task 1}
Let $\omega$ be any eigenvalue of A and let x be an eigenvector corresponding to $\lambda$. Then 
\begin{equation*}
\cfrac{\|Ax\|}{\|x\|}=\cfrac{\|\lambda x\|}{\|x\|}=\cfrac{ |\lambda|\cancel{\|x\|}}{\cancel{\|x\|}}=\lambda.
\end{equation*}
Hence $|\lambda| \le \|A\|$ for all eigenvalues and in particular,
\begin{equation*}
\rho\left(A\right) \overset{\text{def.}}{=} \overset{\rho}{\underset{i=1}{\max}}|\lambda| \le \|A\|.
\end{equation*}

\subsection*{Task 2}
From $\|AB\| \le \|A\|\|B\|$, it follows that $\|A^n\|\le \|A\|^n \rightarrow 0$. 

Hence $\lim\limits_{n\rightarrow \infty }A^n=0$.

\subsection*{Task 3}
\begin{itemize}
	\item [1.]
	\begin{align*}
	\|x\|_\infty&=\underset{i\le n}{\max} |x_i|\\
	&=|x_i|\\
	&=\sqrt{x_j^2}\\
	&=\sqrt{x_1^2 + x_2^2+...+x_n^2}
	\end{align*}
	
	\item [2.]
	\begin{align*}
	\|x\|_2&=\sqrt{\sum_{i=1}^n x_j^2}=\sqrt{n x_j^2}\\
	&\le\sqrt{n}|x_j|\\
	&=\sqrt{n}\|x\|_\infty.
	\end{align*}
	\item [3.]
	\begin{align*}
	\|A\|_\infty &=\underset{\|x\|=1}{\max} \|Ax\|_\infty \\
	&=\underset{\|x\|=1}{\max} \|Ax\|_2\\
	&=\|A\|_2\le \sqrt{n}\|A\|_2.
	\end{align*}
	
	\item [4.]
	\begin{align*}
	\|A\|_2&=\underset{\|x\|=1}{\max}\|Ax\|_2 \\
	&\le \underset{\|x\|=1}{\max} \sqrt{n}\|Ax\|_\infty \\
	&= \|Ax\|_\infty.
	\end{align*}
\end{itemize}



\subsection*{Task 4}
We have that 
\begin{equation*}
\cfrac{\|Ax\|_2}{\|x\|_2}\le \|A\|_2,
\end{equation*}
for all $x\in \mathbb{R}^n$.
\begin{equation*}
\cfrac{\|Aa\|_2}{\|a\|_2}=\cfrac{\|\sqrt{\sum_{i=1}^{n} a_i^2}\|_2}{\sqrt{\sum_{i=1}^{n} a_i^2}}=\sqrt{\sum_{i=1}^{n} a_i^2}= \|a\|_2,
\end{equation*}
whence, $\|a\|_2\le \|A\|_2$.
We have also that
\begin{equation*}
\cfrac{\|Ax\|_2}{\|x\|_2}=\cfrac{\|a^Tx\|_2}{\|x\|_2}\overset{\text{C-S}}{\le} \cfrac{\|a\|_2 \|x\|_2}{\|x\|_2}.
\end{equation*}
Hence, $\|A\|_2= \|a\|$.

Now we will try to show the same theorem but for the 1-norm.
\begin{align*}
\|A\|_1&=\underset{x\in S^n}{\max} \|Ax\|_1\\
&=\underset{x\in S^n}{\max} \left| \sum_{i=1}^{n}a_i x_i\right|\\
&\le\underset{x\in S^n}{\max} \sum_{i=1}^{n}|a_j||x_i|\\
&\le\underset{1\le i \le n}{\max}|a_i|.
\end{align*}
And we also have that
\begin{equation*}
\|Ae_i\|_1=|a_i|\le \|A\|_1.
\end{equation*}
Hence, $ \|A\|_1=\underset{1\le i \le n}{\max} |a_i|$. And now set $A= \|[1 2]\|$ and $a=\|(1,2)\|$, whence $\|A\|_1\not= \|a\|_1$.
\subsection*{Task 5}
Let $\varepsilon >0$ be given. We want $ \left| \|x\|_B-\|y\|_B\right|<\varepsilon $ whenever $\|x-y\|_1<\delta $, for some $\delta >0$. From the reverse triangle inequality we get that 
\begin{equation*}
\|x-y\|<\delta \Leftrightarrow \sum_{i=1}^{n} |x_i-y_i|<\delta,
\end{equation*}
whence $|x_i-y_i|<\delta$ for all $i$.
We have also that $\|e_i\|=r_i$, and
\begin{equation*}
\|x\|_B=\|\sum x_i e_i\|_B\le \sum_{i=1}^{n} r_i |x_i|.
\end{equation*}
So we get that,
\begin{equation*}
\|x-y\|_B\le \sum_{i=1}^{n} r_i|x_i-y_i|=R\delta.
\end{equation*}
\subsection*{Task 6}


\subsection*{Task 7}

TBD

\end{document}
