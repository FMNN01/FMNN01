\documentclass[a4paper,12pt]{article}

\usepackage{amsfonts}
\usepackage{amsmath}
\usepackage{amssymb}
\usepackage{amsthm}

\usepackage[english]{babel}
\usepackage[T1]{fontenc}
\usepackage[utf8]{inputenc}


\newcommand{\argmax}{\mathrm{argmax}}
\newcommand{\rank}{\mathrm{rank}}

\newtheorem{proposition}{Proposition}



\begin{document}



\title{Homework 5 \\ NUMA11}
\author{
  Karl \textsc{Lind\'{e}n} \\
  <karl.linden.887@student.lu.se> \\
  Oscar \textsc{Nilsson} \\
  <erik-oscar-nilsson@live.se>
}

\maketitle
\thispagestyle{empty}

\newpage


\subsection*{Task 1}

TBD


\subsection*{Task 2}

Let \(A \in \mathbb{C}^{m \times m}\) be tridiagonal with all its superdiagonal
entries are non-zero.
Then
\[
  A =
    \begin{pmatrix}
      a_1    & b_2    & 0      & \cdots & 0       & 0       & 0      \\
      c_1    & a_2    & b_3    & \cdots & 0       & 0       & 0      \\
      0      & c_2    & a_3    & \cdots & 0       & 0       & 0      \\
      \vdots & \vdots & \vdots & \ddots & \vdots  & \vdots  & \vdots \\
      0      & 0      & 0      & \cdots & a_{m-2} & b_{m-1} & 0      \\
      0      & 0      & 0      & \cdots & c_{m-2} & a_{m-1} & b_m    \\
      0      & 0      & 0      & \cdots & 0       & c_{m-1} & a_m
    \end{pmatrix}
\]
where \(b_k \ne 0\) for all \(k\).
Let
\[ f: \mathbb{C} \ni x_1 \mapsto (x_1,x_2,\dots,x_m) \in \mathbb{C} \]
where the components are defined recursively by
\[
  x_2 = - \frac{a_1 x_1}{b_2}
  \quad \text{and} \quad
  x_k = - \frac{c_{k-2}x_{k-2} + a_{k-1}x_{k-1}}{b_k}, \enspace 3 \le k \le m.
\]

By induction it can easily be shown that \(f\) is linear, although this is a
very tedious task.
Therefore we leave the proof of this to the reader.

Now suppose \(x = (x_1, x_2, \dots, x_m)\) is a vector in the null space,
\(\nspace(A)\), of \(A\).
Then it holds that
\[
  \begin{cases}
                      a_1 x_1         + b_2 x_2 &= 0, \\
    c_1 x_1         + a_2 x_2         + b_3 x_3 &= 0, \\
                                                &\vdots \\
    c_{k-2} x_{k-2} + a_{k-1} x_{k-1} + b_k x_k &= 0, \\
                                                &\vdots \\
    c_{m-2} x_{m-2} + a_{m-1} x_{m-1} + b_m x_m &= 0. \\
  \end{cases}
\]
By an induction argument it is seen that \(x = f(x_1)\).
This shows that
\[ \nspace(A) \subseteq f(\mathbb{C}) \]
and because \(f(\mathbb{C})\) is a one-dimensional subspace of \(\mathbb{C}^m\),
it follows that \(\nspace(A)\) is atmost one-dimensional.
An application of the dimension theorem gives that \(\rank(A) \ge m - 1\) for
any tridiagonal complex matrix whose superdiagonal entries are non-zero.

Now let \(A\) be tridiagonal and hermitean with non-zero superdiagonal entries.
By applying the previous result to \(A - \lambda I\) we have that
\[ \rank(A - \lambda I) \ge m-1 \]
for all \(\lambda \in \mathbb{C}\).
Lastly, because any hermitean matrix has an orthogonal basis of eigenvectors it
follows that the geometric and algebraic multiplicity of the eigenvalues are
equal, but because all eigenvalues have multiplicity one, it also follows that
every eigenvalue has algebraic multiplicity one.
Thus, the eigenvalues are distinct.


\begin{comment}
The function \(f\) is linear, as is now shown with induction.
It shall be shown that
\begin{equation}\label{eq:f-linear-kth-component}
  (f(sx_1 + ty_1))_k = (sf(x_1) + tf(y_1))_k
\end{equation}
for all \(s, t, x_1, y_1 \in \mathbb{C}\) and \(1 \le k \le m\).
From this it then follows that
\[ f(sx_1 + ty_1) = sf(x_1) + tf(y_1) \]
i.e. that \(f\) is linear.
Firstly,
\[ (f(sx_1 + ty_1))_1 = sx_1 + ty_1 = (sf(x_1) + tf(y_1))_1. \]
Secondly,
\begin{align*}
  (f(sx_1 + ty_1))_2
    &= -\frac{a_1(sx_1 + ty_1)}{b_2} \\
    &= s\left(-\frac{a_1x_1}{b_2}\right) + t\left(-\frac{a_1y_1}{b_2}\right) \\
    &= (sf(x_1) + tf(y_1))_2
\end{align*}
Suppose that \eqref{eq:f-linear-kth-component} holds for \(k=l-2\) and
\(k=l-1\).
Then one has
\end{comment}


\subsection*{Task 3}

TBD


\end{document}
