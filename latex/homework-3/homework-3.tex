\documentclass[a4paper,12pt]{article}

\usepackage{amsfonts}
\usepackage{amsmath}
\usepackage{amssymb}
\usepackage{amsthm}

\usepackage[english]{babel}
\usepackage[T1]{fontenc}
\usepackage[utf8]{inputenc}


\newcommand{\argmax}{\mathrm{argmax}}
\newcommand{\rank}{\mathrm{rank}}

\newtheorem{proposition}{Proposition}



\begin{document}



\title{Homework 1 \\ NUMA11}
\author{
  Karl \textsc{Lind\'{e}n} \\
  <karl.linden.887@student.lu.se> \\
  Oscar \textsc{Nilsson} \\
  <erik-oscar-nilsson@live.se>
}

\maketitle
\thispagestyle{empty}

\newpage

\section*{Task 1}

\section*{Task 2}
Let \( A = U \Sigma V^T \) where \(u_j\) is the j:th column of \( 
U\). Now we have 
\[ V \Sigma^{-1}U^T u_j = V \Sigma^{-1} e_j. \]
By taking the norms on each side we have
\[ \|V \sigma^{-1} \| = \| \sigma^{-1} \|, \]
because \( V \) is orthogonal.

Thus, by choosing \( b = u_1\) and \( \delta(x)\) and \(\delta(b) \) 
that 
\begin{align*}
\| x \| &= \frac{1}{\| \sigma_1 \| } \Leftrightarrow\\
\|\delta x\| &= \frac{1}{\| \sigma_n \| } \Leftrightarrow\\
\|b\| &= 1 \Leftrightarrow\\
\|\delta b\| &= 1.
\end{align*}
Plugging this in we have,
\begin{align*}
\frac{\|\delta b\|}{\| x\|} &= \frac{\|\sigma_1\|}{\|\sigma_n\|}\\
&= \kappa(A) \frac{\| \delta b\|}{\|b\|}.
\end{align*}
as desired.

\section*{Task 3}

\appendix
\section*{Task 1 and 3}
\lstinputlisting[firstline = 9 
]{../../python/homework-3/homework-3.py}
\end{document}
