\documentclass[a4paper,12pt]{article}

\usepackage{amsfonts}
\usepackage{amsmath}
\usepackage{amssymb}
\usepackage{amsthm}

\usepackage[english]{babel}
\usepackage[T1]{fontenc}
\usepackage[utf8]{inputenc}


\newcommand{\argmax}{\mathrm{argmax}}
\newcommand{\rank}{\mathrm{rank}}

\newtheorem{proposition}{Proposition}



\begin{document}



\title{Homework 4 \\ NUMA11}
\author{
  Karl \textsc{Lind\'{e}n} \\
  <karl.linden.887@student.lu.se> \\
  Oscar \textsc{Nilsson} \\
  <erik-oscar-nilsson@live.se>
}

\maketitle
\thispagestyle{empty}

\newpage


\subsection*{Task 1}

The assumption that
\[
  A = \begin{pmatrix}
    a_{11} & w^T \\
    w      & A_1
  \end{pmatrix}
\]
is positive definite means that
\[ x^T A x \ge 0 \]
for all \(x \in \mathbb{R}^n\) with equality if and only if \(x = 0\).
In particular
\[
  e_1^T A e_1
    = e_1^T
      \begin{pmatrix}
        a_{11} \\
        w
      \end{pmatrix}
    = a_{11}
    > 0
\]
where \(e_1 = (1,0,0,\dots,0)\).

Now consider any vector \(y \in \mathbb{R}^{n-1}\) and let
\[
  x = \begin{pmatrix}
    0 \\
    y
  \end{pmatrix}.
\]
We now have that
\[
  x^T A x
    =
      \begin{pmatrix}
        0 & y^T
      \end{pmatrix}
      \begin{pmatrix}
        a_{11} & w^T \\
        w      & A_1
      \end{pmatrix}
      \begin{pmatrix}
        0 \\
        y
      \end{pmatrix}
    =
      \begin{pmatrix}
        0 & y^T
      \end{pmatrix}
      \begin{pmatrix}
        w^T y \\
        A_1 y
      \end{pmatrix}
    = y^T A_1 y
    \ge 0
\]
with equality if and only if \(x = 0\), but this happens if and only if
\(y = 0\), showing that \(A_1\) is positive definite.


\subsection*{Task 2}

Let \(A = (a_{ij})\) be strictly diagonally dominant and suppose
\[ Ax = 0 \]
where \(x = (x_j) \ne 0\).
This implies that
\[ \sum_{j=1}^n a_{ij} x_j = 0 \]
for all \(i = 1, \dots, n\).
By the assumption that \(x \ne 0\) there is an \(i_1\) such that
\(x_{i_1} \ne 0\).
For a fixed \(i_1\) it follows that
\[ \sum_{j \ne i_1} a_{i_1j} x_j = -a_{i_1i_1}x_{i_1} \]
By the assumption on \(A\) and the triangle inequality we have
\[
  \sum_{j \ne i_1} |a_{i_1j}| |x_j|
    \ge \left| \sum_{j \ne i_1} a_{i_1j} x_j \right|
    = |a_{i_1i_1}| |x_{i_1}|
    > \sum_{j \ne i_1} |a_{i_1j}| |x_{i_1}|.
\]
In particular there is an \(i_2 \ne i_1\) such that \(|x_{i_2}| > |x_{i_1}|\).
Repeating this process \(n\) times we get a finite sequence
\(i_1, i_2, \dots, i_{n+1}\) that fulfills
\[ |x_{i_1}| < |x_{i_2}| < \dots < |x_{i_n}| < |x_{i_{n+1}}|. \]
By the pigeonhole principle \(i_r = i_s\) for some \(r < s\).
Now we arrive at the contradiction
\[ |x_{i_r}| < |x_{i_s}| = |x_{i_r}|. \]
This means the assumption that \(x \ne 0\) is false and consequently that
\(Ax = 0\) only has the trivial solution.
Hence, \(A\) is invertible.


\subsection*{Task 4}

TBD


\subsection*{Task 5}

TBD


\end{document}
